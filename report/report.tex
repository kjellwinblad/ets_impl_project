%\documentclass[aps,pre,preprint,groupedaddress,nofootinbib]{revtex4}
\documentclass[aps,pre,preprint,nofootinbib]{revtex4}
%\documentclass[aps,twocolumn,pre,nofootinbib]{revtex4}   % list options between brackets
\usepackage{}              % list packages between braces

% type user-defined commands here

\begin{document}

\title{Erlang Term Storage Implementation}
\author{Kjell Winblad and Stavros Aronis}
\date{\today}


\begin{abstract}

  This report describe in detail how the Erlang Term Storage (ETS) is implemented.
  The report is created in the RELEASE project as the first step towards creating a more scalable ETS implementation.
  ETS tables are common way to share data between processes in Erlang programs.
  It is important to have a very good ETS implementation because shared data is often the bottleneck in concurrent applications.
  The main goals of the report are:
  \begin{itemize}
   \item to get an understanding of the current implementation of the different ETS tables
   \item and to communicate this knowledge to for example the Erlang OTP team. 
  \end{itemize}

\end{abstract}

\maketitle

\section{Introduction}

Erlang Term Storage (ETS) tables is a part of Erlang standard library. It is used as a way to efficintly share memory between processes. An Erlang ETS table share many properties with ordinary Erlang processes. The programer can manage ETS tables by a set of  An ETS table is identified by an identifier that is created 

\section{Handling of Tables}

The infrastructure for the hanling of tables is described in this secrion. An overview of the datastructures involved is provided in section~\ref{sec:tables_overview}. How locking is done when tables are accessed concurrently is described in section~\ref{sec:tables_locking}.

\subsection{Overview} \label{sec:tables_overview}



\subsection{Locking} \label{sec:tables_locking}


\section{Different Table Types}

\subsection{Set}     % section 2.1
\subsubsection{Data Structure}
\subsubsection{Expanding}
\subsubsection{Shrinking}

\subsection{Bag}
\subsection{Duplicate Bag}
\subsection{Ordered Set}


\section{Concurrency Options}

\subsection{Write Concurrency}

\subsection{Read Concurrency}


\section{Benchmark}

\subsection{Description}

\subsection{Results}


\section{Scalable ETS Suggestions}

\begin{thebibliography}{9}
  % type bibliography here
\end{thebibliography}

\end{document}
